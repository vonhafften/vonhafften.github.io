\documentclass{beamer}
\usetheme{Boadilla}
\usepackage{graphicx}

\title[Optimal Risk Weights]{Optimal Risk Weights}
\author{Alex von Hafften}
\institute{UW-Madison}

\begin{document}

\begin{frame}
\titlepage
\end{frame}

\begin{frame}
\frametitle{Motivation}
\begin{itemize}[<+->]
\item Banks create short-term, safe, and liquid liabilities (e.g. deposits) from long-term, risky, and illiquid assets (e.g. loans) (Gorton and Pennacchi 1990).
\bigskip
\item Moral hazard (Kareken and Wallace 1978):
\bigskip
\begin{itemize}[<+->]
\item \textbf{Distorted incentives} from deposit insurance (i.e. limited liability) and expectations of ex-post bailout (e.g. too big to fail).
\bigskip
\item \textbf{Asymmetric information:} Regulators and depositors have less information about the riskiness of bank assets.
\end{itemize}
\bigskip
\item Moral hazard leads banks to take on excessive credit risk resulting in bank failures hampering credit availability, financial stability, and economic activity (Romer and Romer 2017).
\bigskip
\item \textbf{Broad research question:} How should bank regulation address moral hazard?
\end{itemize}
\end{frame}

\begin{frame}
\frametitle{Risk-Based Capital Requirements}
\begin{itemize}[<+->]
\item To address moral hazard, banks are subject to risk-based capital (RBC) requirements: 
$$
E \ge \mathbf{A} \cdot \mathbf{w}
$$
\begin{itemize}[<+->]
\item $E$ is shareholder equity (i.e. ``capital") in the bank.
\item $\mathbf{A}$ is a vector of bank assets.
\item $\mathbf{w}$ is a vector of risk weights.
\end{itemize}
\bigskip
\item The higher credit risk of asset $A_i$, the higher $w_i$.
\begin{itemize}[<+->]
\item E.g. $w_{\text{Treasury}} = 0$, $w_{\text{residential mortgage}} \approx 0.5$, and $w_{\text{corporate debt}} \approx 1.0$.
\end{itemize}
\bigskip
\item A bank invested primarily in Treasuries and mortgages has lower risk weighted assets and thus is required to hold less capital than a com­parable bank invested primarily in corporate debt.
\bigskip
\item RBC requirements aim to address moral hazard by forcing banks to have ‘skin in the game’ and internalize the social costs of bank failures.
\end{itemize}
\end{frame}

\begin{frame}[label=risk_weights]
\frametitle{How To Calculate Risk Weights?}
\begin{itemize}[<+->]
\item Under Basel I (1988), ``less informed" regulators set $\mathbf{w}^R$. \hyperlink{basel_i}{\beamerbutton{More}}
\begin{itemize}
\item \textbf{Benefit:} Simple and transparency. 
\item \textbf{Cost:} Coarse risk weights led to distortions in credit allocation within bucket (Jones 2000).
\end{itemize}
\bigskip
\item Under Basel II (2006), ``more informed" banks could determine $\mathbf{w}^B$ using in-house models. \hyperlink{basel_ii}{\beamerbutton{More}}
\begin{itemize}
\item \textbf{Benefit:} Banks have better information about their riskiness. 
\item \textbf{Cost:} Bank have an incentive to underestimate risk.
\item Behn, Haselmann, and Vig (forthcoming) find evidence of banks gaming risk weights. \hyperlink{bhv}{\beamerbutton{More}}
\end{itemize}
\bigskip
\item Under Basel III (2015), banks use $\max\{\mathbf{w}^R, \mathbf{w}^B\}$.
\begin{itemize}
\item Possibly a blunt way to address gaming by banks?
\end{itemize}
\end{itemize}
\end{frame}



\begin{frame}
\frametitle{Research Question}
\begin{itemize}[<+->]
\item \textbf{Key tradeoff:} Banks have better information about their riskiness, but they have an incentive to underestimate risk.
\bigskip
\item \textbf{Specific research questions:} 

\bigskip

Given this trade-off between information and incentives,

\bigskip

\begin{itemize}[<+->]
\item How do rules about risk weights change probability of bank failure and lending?
\bigskip
\item What are optimal risk weights?
\end{itemize}
\end{itemize}
\end{frame}




\begin{frame}
\frametitle{First Steps}


Develop a two-period model (in the spirit of Holmstrom and Tirole 1997, Kareken and Wallace 1978, and Townsend 1979):

\bigskip

\begin{itemize}[<+->]

\item A bank with deposit insurance (limited liability).

\bigskip

\item The bank can fund a risky project and can engage in costly, unobservable monitoring to increase the probability of success.

\bigskip

\item RBC requirement establishes minimum amount of internal equity financing.

\bigskip

\item Explore risk weights akin to Basel I, II, and III.

\bigskip

\item Extend to dynamic setting to incorporate reputation effects (in the spirit of Dovis and Kirpalani 2020).

\end{itemize}
\end{frame}



\begin{frame}[label = basel_i]
\frametitle{Basel I Risk Weights}

\begin{center}
\begin{tabular}{ r | l}
Risk Weight (\%) & Asset types\\ 
\hline
  0 & Cash, bullion, Treasuries \\  
 20 & MBS with AAA rating \\
 50 & Municipal bonds, residential mortgages\\
100 & Corporate debt
\end{tabular}
\end{center}

\bigskip

\textit{The framework of weights has been kept as simple as possible and only five weights are used.} 

\begin{center}
 Basel (1988)
\end{center}

\hyperlink{risk_weights}{\beamerbutton{Back}}


\end{frame}


\begin{frame}[label = basel_ii]
\frametitle{Basel II Risk Weights}

\begin{itemize}[<+->]


\item The primary motivation for Basel II was to achieve greater sensitivity to credit risk across assets (Gordy and Heitfield 2012) using a ``standardized approach" and an ``internal-ratings based approach"

\bigskip

\item ``Standardized approach" 

\begin{itemize}[<+->]

\item Similar to Basel I but with finer risk weight buckets.

\end{itemize}


\bigskip

\item ``Internal-ratings based approach"

\begin{itemize}[<+->]

\item Under the ``internal-ratings based approach", banks estimate borrow-specific default probabilities and loans to borrowers with higher default probabilities receive higher risk weights (Behn, Haselmann, and Vig forthcoming).

\bigskip

\item Risk weights are computed using a variety of credit risk models (e.g. some banks run over 100 different models).

\bigskip

\item These models are designed and calibrated by banks and then the estimates are approved by the bank supervisor.

\end{itemize}
\end{itemize}


\hyperlink{risk_weights}{\beamerbutton{Back}}

\end{frame}


\begin{frame}[label = bhv]
\frametitle{Behn, Haselmann, and Vig (forthcoming)}


\begin{itemize}

\item Use loan-level data from Germany to study the introduction of capital requirements using internal-rating based risk weights.

\bigskip

\item They find that banks systematically underreported risk.

\bigskip

\item Banks with higher gains from underestimating risks underestimate risks more.

\bigskip

\item Larger banks benefit more than smaller banks.

\end{itemize}

\hyperlink{risk_weights}{\beamerbutton{Back}}

\end{frame}


\begin{frame}
\frametitle{References}
\scriptsize

Basel Committee on Banking Supervision (1988). ``International Convergence of Capital Measurement and Capital Standards." https://www.bis.org/publ/bcbs04a.pdf

\bigskip


Basel Committee on Banking Supervision (2006). ``Basel II: International Convergence of Capital Measurement and Capital Standards: A Revised Framework." https://www.bis.org/publ/bcbs128.htm

\bigskip

Behn, Markus, Rainer Haselmann, and Vikrant Vig (forthcoming). ``The Limits of Model-Based Regulation." Journal of Finance.

\bigskip

Bernanke, Ben (1983). ``Nonmonetary Effects of the Financial Crisis in the Propagation of the Great Depression." American Economic Review Vol. 73 (3).

\bigskip

Gordy, Michael and Eric Heitfield (2012). ``Risk-Based Regulatory Capital and Basel II," The Oxford Handbook of Banking (1 ed.) Edited by Allen N. Berger, Philip Molyneux, and John O. S. Wilson.

\bigskip

Gorton, Gary, and George Pennacchi (1990). ``Financial Intermediaries and Liquidity Creation," Journal of Finance Vol XLV (1).

\bigskip

Romer, Christina, and David Romer ``New Evidence on the Aftermath of Financial Crisis is Advanced Countries." American Economic Review 107 (10).

\bigskip

\end{frame}

\end{document}

